\documentclass[../../templates/section]{subfiles}

\begin{document}

\section{Finite, Countable, and Uncountable Sets}\label{sec:finite-countable-and-uncountable-sets}

Armed with a well-constructed real number system, we can begin to talk about
some mathematical constructions. One of the simplest, and some might say most 
natural, constructions on the real numbers is a function. A function
is a rule that assigns an input number and outputs a number. 

However, the notion of assigning an output to a given input still makes sense 
even we do not consrain ourselves to them being numbers. In fact, a 
function can be defined on an arbitrary set $A$, where ``inputs" are taken from t
he elements of $A$. While the ``outputs" may be elements of some set $B$. F
or example, we may define a function on the set
of names of people in the class. The function outputs the birthday of the person in
binary. In this case. To make things clear, we will usually, unless other
wise noted, call the type of 
``general functions'' between arbitrary sets a \empth{mapping}, and reserve t
he term \emph{function} for 
functions in the sense of mappings between sets of real numbers. However, o
ne should keep in mind that
in mathematics the two terms are often used interchangeably.

In our example, we would call the set of names, the set $A$, the \emph{domai
n} of the function, and
the set of all possible dates the \emph{range} of the function, and we say the
function maps from the domain to the range. If we denote $f$ as the funciton,
the domain as the set $A$, the range as the set $B$, then we represent this as
\[
f:A\to B
\]
Now one might notice that in a typical class, it is unlikely that we will
have all the possivle dates in a year represented among all the students,
unless this is an unusially large class.
Therefore, not all of the elements of $B$ are ``hit'' by the funciton. If it is
the case we \emph{do} ``hit'' all of $B$, then we call such a function a surjection
(or an \emph{onto} function). This is made more precise in the following definition:

\begin{definition}
A mapping $f:A\to B$ is said to be a \emph{surjection}, or a \emph{surjective
mapping}, if for all $y\in B$, there exists an element $x\in A$, such that
$f(x) = y$.
\end{definition}

Sometimes, we are interested in talking about the values of a particular 
subset of the domain under the mapping. For example, we may only wish to observe the
birthdays of males in the classroom. In this case, suppose $E \subset A$ is a
subset of $A$, then we say $f(E)$ is the \emph{image of $E$ under $f$}, where $f(E)$
consists of all the values of the elements of $E$ when mapped by $f$:
\[
	f(E) \coloneqq \{y\in B: f(x) \text{ for some }x\in A}\}

	\]

With this notion we can alternatively define a surjective mapping as one where
$f(A) = B$ (a non-surjective mapping would only satisfy $f(A) \subset B$). We call $f(A)$ the 
 \emph{image of the mapping $f$}, or just \emph{image of $f$}.

On the other hand, we can fix some subset of the range $B$, and consider the
elements in $A$ that ``hit'' $B$. To make it precise, if $G \subset B$, then we
denote $f^{-1}(E)$, the inverse image of $G$, as the set of all $x\in A$, such
that $f(x)\in G$. It is entirely possible that multiple people may have the
same birthday, so given an element $y\in B$, the set $f^{-1}(y)$, that is all
$x\in A$ that map to $y$ may have multiple elements. However, it is not
possible for a person to have multiple birthdays; in the same way, for any
$x\in A$, $f(x)$ is a single value. This is inherent in the definition of a
function.

In the special case that we do not have more than one element of the domain 
mapping to the same element of the range, i.e., we have noone sharing the same
birthday, then we call this function an \emph{injection}. More precisely:

\begin{definition}
	A mapping $f:A\to B$ is said to be an \emph{injection}, or an \emph{injective
	mapping}, if for all $x_1, x_2 \in A$, whenever it is the case that $x_1\neq
x_2$, we have $f(x_1) \neq f(x_2)$.
\end{definition}

The definition says that, whenever we have distinct $x$'s in the domain, this
must mean that they map to different $y$'s in the domain. This is exactly what
we meant by no distinct people from the class have the same birthday. 

The contrapositive of this definition, which is sometimes useful is that a mapping
is an injection precisely whenever $y\neq y'$, where $y,y'$ are in the image of $f$, we
 have $f^{-1}(y) \neq f^{-1}(y')$.

Now if we have a mapping that is both surjective and injective, we call such a
mapping a \emph{bijection}, or a \emph{bijective mapping}.

\begin{definition}
A mapping $f:A\to B$ is said to be a bijection, or a bijective mapping,
if it is both a surjection, and an injection.
\end{definition} 

Intuitively, a bijection is a mapping that does not miss anything in the range
(surjective); and assigns each distinct $x$ from the domain, to a distinct and
unique $f(x)$ (injective). As such, a bijection is sometimes called a \emph{one-to-one 
correspondence}.


The following diagrams illustrate a surjection, injection, and a bijection.

The existence of a bijective mapping between two sets is useful since we are
assured that the two have the same number of elements.  (you should convince
yourself why this is the case). In fact, this is exactly how we properly
compare the number of elements between sets. If we can find a bijective mapping
between set $A$ and set $B$, then they have the same number of elements; if
there exist no bijective mapping, then the two must have diff erent number of
elements. 

In set theoretic terms, two sets have the same cardinality, or the same
cardinal number, if they have the same number of elements.

\begin{definition}
If there exists a bijective mapping between sets $A$ and $B$, we say they are 
equivalent. And we write $A \sim B$.
\end{definition}

Thus, we have a particular way of comparing two sets. If we are given two sets,
we can have a well-defined notion of what we mean when we say ``they have the
same number of elements'', or ``they do not have the same number of elements''.
A ``comparison'' such as the one described, gives us a ``relation'' between
sets. $A$ and $B$ are equivalent, if they have the same cardinality

OK THIS PART I HAVENT FIGURED OUT A WAY TO DO IT, IT'S ON EQUIV. RELATIONS

With a well-defined way of comparing the number of elements of a set, we can
begin to define what it means to have finite, or infinite, number of elements.

Intuitively, if a set were to have a finite number of elements, we ought to be 
able to count the elements one by one, and finish at some number $n$, which we
call the ``number of elements that this set has''. 

What exactly are we doing when we ``count'' the elements of a set $A$? We are
grabbing elements out of $A$, and tagging it with a unique number ($1, 2, 3, 4,
5, \ldots$), and throwing the ``element-number'' pair into the ``counted''
bucket. We continue until we run out of elements, and then we take the number
on the last pair, and call that the cardinality, or the number of elements.
This ``counting'' operation is precisely a bijective mapping between $A$, and
the counting numbers, or Natural Numbers, which are $1, 2, 3, 4, 5, 6, 7,
\ldots$. And when we say set $A$ has cardinality $n$, we mean precisely the
following:\todo{@kevin, notice that you use quite a bit of quotes in the past
few paragraphs. perhaps some italicized? also learn how to do quotes in latex}
\[
A \sim \{1, 2, 3, 4, 5,\ldots, n\}
\]
The existence of such an $n$, that is to say, our counting terminates, is
exactly the definition of a set being finite.  While the absence of such an $n$
is exactly the definition of a set being infinite. 

We summarize these defininitions this below. In addition, we hint at the
existence of sets which are infinite but are not equiavlent to the set of all
natural numbers. That is to say there are sets which cannot form a bijection
with some set $\{1, 2, \ldots, n\}$ but also cannot form a bijection with the
set $\{1, 2, 3, 4, \ldots\}$.  (This might seem like a big surprise, but think
back to Chapter 1, when we convinced you that the rational numbers is not a
``big'' enough number system)

\begin{definition}
For simplicity, we write the set $\{1, 2, 3,\ldots, n\}$ as $J_n$\todo{i really
dont like this notation. I would prefer $\overline{n}$ or something else}, and
the set $\{1, 2, 3, 4, \ldots\}$ as $\N$. Then for any set $A$, we say
\begin{itemize}
    \item $A$ is finite if $A \sim J_n$ for some $n$ (the empty set is also
        considered to be finite, even though technically it has zero elements).
    \item $A$ is infinite if $A$ is not finite.
    \item $A$ is countable if $A \sim J$
    \item $A$ is uncountable if $A$ is neither finite nor countable. 
    \item $A$ is at most countable if $A$ is finite or countable.
\end{itemize}
\end{definition}

Now for some examples. In particular, we will show the (possibly) surprising
result that sets of all integers, the set of only positive integers (which is
just $\N$), the set of even integers, and the set of odd integers all have the
same cardinality, the countably infinite kind.  That is to say, they all have
the same number of numbers in them.  We will do this by explicitly defining
bejective mappings between each set to $\N$.

\begin{example}
$\Z = \{\ldots, -3, -2, -1, 0, 1, 2, 3, \ldots\}$ is the set of all integers.
$\Z$ is countable, i.e., $\Z \sim \N$. The bijection $f:\N \to \Z$ is
given by the following
\[ 
f(n) =
\begin{cases} 
    \frac{n}{2}      & \text{if} \quad n \quad \text{is even} \\
    -\frac{n - 1}{2} & \text{if} \quad n \quad \text{is odd}
\end{cases}
\]
We can check that this is indeed a bijection between $\N$ and $\Z$.  To do so
we show that it is both a surjection and an injection. First, for any natural
number $y\in \Z$, if $y$ is positive, then we can find the integer $x = 2y$
which is an even natural number, thus under the mapping,
\[
f(x) = f(2y) = \frac{2y}{2} = y
\]
If $x$ is not positive (so is 0 or is negative), then we can find the integer
$y = -2y + 1$, which is an odd natural number, therefore under the mapping,
\[
f(x) = f(-2y + 1) = -\frac{-2y + 1 - 1}{2} = -\frac{2y}{2} = y
\]
So any integer $y$ in $\Z$, we have found a natural number in $\N$ that is
mapped by $f$ to $y$, therefore $f$ is surjective.

Now for injectivity. Suppose we take two disitinct natural numbers, $x_1$ and
$x_2$, $x_1\neq x_2$. If they have different parities, then the mapping $f$
maps them to different signs, therefore $f(x_1)\neq f(x_2)$. If they are both
even, then since $x_1\neq x_2$, we must have $f(x_1) = \frac{x_1}{2}\neq
\frac{x_2}{2} = f(x_2)$. If $x_1$ and $x_2$ are both odd, then similarly we
have $f(x_1) = -\frac{x_1-1}{2}\neq -\frac{x_2-1}{2} = f(x_2)$. Therefore we
have established that $f$ is an injection. Since we showed both surjectivity
and injectivity, $f$ is a bijection. Therefore, by our definition of
cardinality, $\N$ and $\Z$ have the same cardinality, i.e., they have the same
number of elements.

The cases for finding a bijection between $mathbb{N}$ and the set of even
integers, and between $\N$ and odd integers, are left as excercises for you.
The general appraoch is the same as the case shown just now.
\end{example} 

\begin{remark}
Recall that a proper subset of a set $A$ is a subset of $A$ that is not the
whole of $A$.  It is not possible for a proper subset of a finite set to be
equivalent (same cardinality) to the set itself. This is intuitively apparent,
a subset that is not equal to the original set must have less elements.
However, if the original set is an infinite set, then this is possible. As we
have just shown, $\N \sim \Z$, but $\N$ is a proper subset of $\Z$.
\end{remark}
In fact, using this logic, we can equivalently say that infinite sets are 
just the sets which we can find a proper subset that is equivalent to the 
set itself.

Now, the usefulness of having properly-defined notions of mappings, and notions 
of cardinality, can be first exemplified by talking about sequences of numbers. 
The definition will appear to be surprisingly barebones. 

\begin{definition}
A \textbf{sequence} is a function on the set of natural numbers $\mathbb{N}$
\end{definition}

An example of a sequence might be the function $f:\mathbb{N}\rightarrow \mathbb{R}$, 
defined by 
\[
f(x) = \frac{1}{x}
\]
For any $x\in \mathbb{N}$.

Then the values of this sequence are the numbers $1, \frac{1}{2}, \frac{1}{3}, 
\frac{1}{4}, \dots$. 

Even though sequences of numbers are the ones that we will work with most of 
the ti
me, the definition a sequence does not require it to be a sequence of numbers. 
With this definition, we can talk about a sequence of any arbitrary things. 
In which 
case the sequence will be a mapping from $\mathbb{N}$ to an arbitriry set.

Notice that the sequence is the mapping, while the values that we typically think 
of as the contents of the sequence, are in fact the values of the mapping, 
these are called  ``terms" of the sequence. We can think of the sequence as being 
the set of rules that says ``the first term is 1", ``the second term is $\frac{1}
{2}$", ``the third term is $\frac{1}{3}$, etc. In that way it is clear why the 
function itself is the one we choose to represent the sequence. Also worth noting 
is that the terms of a sequence do not have to be 3blue1brownunique, which is to 
say, the function does not have to be injective. 

Although that is the case, in practice, we often identify a sequence by its terms. 
For instance, in our example above, we would denote $f(1)$ as $x_1$, $f(2)$ as 
$x_2$, \dots and $f(n)$ as $x_n$ for any $n$. To denote the whole sequence, 
i.e. the function it self, we write $(x_n)$.

We can easily construct a sequence from any given countably infnite set, since 
for any such set, there must exist a bijective mapping between it and $\mathbb{N}$, 
then there must be a mapping from $\mathbb{N}$ to the set. (either the bijection 
itself or the inverse mapping of the bijection). We describe this property by 
saying that the elements of any countably infnite set can be ``arranged in sequence".puppy
\end{document}
